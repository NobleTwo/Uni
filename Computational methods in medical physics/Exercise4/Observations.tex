\documentclass{article}

\begin{document}
	\section{Mathematical preparations}
	\subsection{f(x)}
		pdf
		\begin{eqnarray*}
			f(x)&=&e^{-x}\\
			x&\epsilon&(0,\infty)
		\end{eqnarray*}
		normalization:
		\begin{eqnarray*}
			\int_0^\infty f(x)\mathrm{dx}&=&\int_0^\infty e^{-x}\mathrm{dx}\\
			&=&-e^{-x}|_0^\infty\\
			&=&1
		\end{eqnarray*}
		cdf
		\begin{eqnarray*}
			F(x)&=&\int_0^x f(t)\mathrm{dt}\\
			&=&\int_0^x e^{-t}\mathrm{dt}\\
			&=&-e^{-t}|_0^x\\
			&=&-e^{-x}+1
		\end{eqnarray*}
		invert:
		\begin{eqnarray*}
			y&=&-e^{-x}+1\\
			\ln(1-y)&=&-x\\
			F^{-1}(y)&=&-\ln(1-y)=x
		\end{eqnarray*}
	\subsection{g(x)}
		pdf
		\begin{eqnarray*}
			g(x)&=&x\cdot e^{-\frac{x^2}{\alpha^2}}, \hspace{0.5cm}\alpha=25^\circ\\
			x&\epsilon&[0^\circ,180^\circ]
		\end{eqnarray*}
		normalization:
		\begin{eqnarray*}
			\int_{0^\circ}^{180^\circ}g(x)\mathrm{dx}&=&\int_{0^\circ}^{180^\circ}x\cdot e^{-\frac{x^2}{\alpha^2}}\mathrm{dx}\\
		\end{eqnarray*}
		substitute $\frac{x^2}{\alpha^2}=:b$, $x\cdot dx=\frac{\alpha^2}{2} db$
		\begin{eqnarray*}
			\int_{0^\circ}^{180^\circ}g(x)\mathrm{dx}&=&\frac{\alpha^2}{2}\int_{0^\circ}^{(180^\circ/\alpha)^2}e^{-b}\mathrm{db}\\
			&=&-\frac{\alpha^2}{2}\left[e^{-b}\right]_0^{(180^\circ/\alpha)^2}\\
			&\approx&\frac{\alpha^2}{2}
		\end{eqnarray*}
		redefine g(x):
		\begin{eqnarray*}
			g(x)&=&\frac{2}{\alpha^2}x\cdot e^{-\frac{x^2}{625^\circ}}
		\end{eqnarray*}
		cdf
		\begin{eqnarray*}
			G(x)&=&\int_0^x g(t)\mathrm{dt}\\
			&=&\frac{2}{\alpha^2}\int_0^x e^{-\frac{				t^2}{625^\circ}}\mathrm{dt}
		\end{eqnarray*}
		substitute $\frac{t^2}{\alpha^2}=:c$, $dt=\frac{\alpha^2}{2} dc$
		\begin{eqnarray*}
			G(x)&=&\int_0^{x^2/\alpha^2} e^{-c}\mathrm{dc}\\
			&=&-\left[e^{-c}\right]_0^{x^2/\alpha^2}\\
			&=&-e^{-x^2/\alpha^2}+1
		\end{eqnarray*}
		invert
		\begin{eqnarray*}
			y&=&-e^{-x^2/\alpha^2}+1\\
			\ln(1-y)&=&-\frac{x^2}{\alpha^2}\\
			x&=&\pm\alpha\sqrt{-\ln(1-y)}\\
			G^{-1}(y)&=&\pm\alpha\sqrt{-\ln(1-y)}
		\end{eqnarray*}
	\section{Computational Setup}
		For implementing the scattering of particles computed by a Monte Carlo method, 6 non-main methods were coded in addition to the main method.\\
		The main-method is used only for initializing the random seed and starting the needed simulation.\\
		For simulating, to different methods were implemented: simulate\_stat() is used for getting the statistic properties of the scattering processes and therefore needs a number of particles that are computed. simulate\_path() computes the path of 5 particles and saves it.\\
		The process for simulating a particle path is the same in both methods: The path is computed by getPath(), if the particle becomes absorbed by becomesAbsorbed() and if it wasn't absorbed the scattering angle by getAngle().\\
		After computation the statistics are saved using write().\\
		
		When computing the path-length, the x-axis is spread by factor 10 to ensure greater accuracy. Therefore, the number of events for a certain path length has to be increased by factor 10 when being compared with f(x).\\
		When computing the scattering angle, via a random number the decision for scattering in positive or negative y-direction is made. Due to this, only half of the expected (by g(x)) particles are measured with a certain scattering angle. Therefore, the statistics are not being compared to g(x) but to $|\frac{g(x)}{2}|$.
	\section{Observations}
	\subsection{Trajectories}
		The particle trajectories show that the particles become absorbed rather quickly. None of the displayed trajectories enters further into the medium than 6[mm].
	\subsection{Path length}
		The path length distribution fits f(x) nicely. As expected, the variance between f(x) and the simulated distribution decreases with increasing number of simulated particles.\\
		A bit strange is the behaviour of the path length for $r=0$: Here, the measured number of particles stay way below (at about half) the expected value. This holds for all tested number of particles.
	\subsection{Absorption ratio}
		The measured absorption ratio is about the exact value of 0.2. With increasing number of particles, again, the variance between expectation and measured values decreases.
	\subsection{Scattering angle}
		The measured scattering angle is distributed almost exactly as given by g(x). As in the above cases, the variance between measurement and expectation decreases with increasing number of sampled particles.
	\subsection{Detector at x=1 and x=5}
		For a detector at x=1 or x=5, a large maximum appears at $y=0$ and two side-maxima at $y\approx\pm0.25$. This effect probably takes place because most events occur directly after inserting the particle at (0/0) with movement exactly along the x-axis. The side-maxima appear due to scattering processes and are therefore distributed as $g(x)$.\\
		The scheme of maxima and side-maxima is clearer to be seen when the number of simulated particles is increased.\\
		It is no surprise that there are more events detected in a detector at x=1 than at x=5 (for $N=100$ even no event at all was registered at x=5) since the second detector is further away from the point of insertion.
\end{document}